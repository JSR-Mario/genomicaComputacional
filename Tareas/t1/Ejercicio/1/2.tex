\item A continuación se presentan 10 secuencias hipotéticas. Diseña una expresión regular que detecte regiones codificantes válidas y específica cuales de las secuencias la cumplen: \vspace{01mm}

\begin{verbatim}
        1.  ATATATACATACTGGTAATGGGCGCGCGTGTGTTAAGTTCTGTTGTAGGGGTGATTAGGGGCG
        2.  GGCCCACACCCCACACCAATATATGTGGTGTGGGCTCCACTCTCTCGCGCTCGCGCTGGGGAT
        3.  ATAAGGTGTGTGGGCGCGCCCCGCGCGCGCGTTTTTTCGCGCGCCCCCGCGCGCGCGCGCGCG
        4.  GGCGCGGGACGCGGCGGCGGATCCCGATCCGTGCGTCAATACTATTATGGCCAGATAGAATAA
        5.  GTGCTGCTGCGGCGCCCACACCTATTATCTCTCTCTCTCTGCCTCTCCACCTCGGGGCTTAAT
        6.  GCGCTGCTGCTGGCTCGATGGGCGCGTGCGTCGTAGCTCGATGCTGGCTCGAGCTGTAATCTT
        7.  GGCGCTCGCTCGGATGCGCGGCCGGGCTCTCTGCTCGCGCTCGCTTCGCGCTCGTGACCGCTG
        8.  AATTGGTGCGCGCTCGCGCACACACAGAGAGAGGGTTTATATAGGATGATATATCCACATTGG
        9.  ATGCTGCTGCTGGCTCTGCTTGCGCTCTGCTCGCTGGGGTGTGTGTGCCGCGCGCTGCTGCTC
        10. GCTGGGCTCGCTCGATGCGCGCGGGCGCGCGACCGCGGACGGCGTCGCTGCTAAATGGGCTTC
\end{verbatim}\vspace{02mm}

Debes entregar una lista con el número de línea en las que hay una región codificante válida (como la hayas revisado en la clase de biología), es decir, si en las líneas 0, 1 y 5 hubiera entonces el resultado debería ser:\vspace{01mm}

\begin{center}
    [0, 1, 5]
\end{center}

