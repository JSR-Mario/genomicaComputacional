\item En el archivo \texttt{promotores.txt} se encuentra la lista de secuencias tomadas del genoma de \textit{Vitis vinifera} y cada una de las secuencias puede que tenga alguno de las diferentes formas en las que se ha encontrado el promotor GATA:

\[
\{AGAT AG, T GAT AG, AGAT AA, T GAT AA\}     
\]

Deseamos estudiar estas regiones en función del promotor GATA y por lo tanto lo primero que deseamos es saber cuántas veces aparecen los promotores en cada región. 

\begin{itemize}
    \item Haz un boxplot con la distribución de cada uno de los promotores

    \item ¿Cuál es la media y desviación estándar de cada promotor?
\end{itemize}

