\item[c)] Entre marzo y junio de 2021 se tuvo un promedio de nuevos contagios semanales de alrededor de 3000 casos, por lo que a lo largo de dos semanas se tendría una prevalencia aproximada de 6000 casos activos respecto a 120000000 de habitantes. Calcula las probabilidades referidas en los dos incisos anteriores pero considerando este nuevo dato de prevalencia. ¿Qué puedes concluir respecto a las probabilidades obtenidas en ambos escenarios?, ¿consideras que en el caso de las pruebas de detección de COVID es necesaria una mayor sensibilidad o una mayor especificidad? Justifica tu respuesta.